% ..............................................................................
% Demo of the fau-beamer template.
%
% Copyright 2022 by Tim Roith <tim.roith@fau.de>
%
% This program can be redistributed and/or modified under the terms
% of the GNU Public License, version 2.
%
% ------------------------------------------------------------------------------
\documentclass[final]{beamer}

% ========================================================================================
% Theme: inner, outer, font and colors
% ----------------------------------------------------------------------------------------
\usepackage[institute=FAU,
			%ExtraLogo = template-art/FAUWortmarkeBlau.pdf,
			%WordMark=None
			aspectratio=169,
			size=18
		   ]{styles/beamerthemefau}
% ----------------------------------------------------------------------------------------
% Input and output encoding
\usepackage[T1]{fontenc}
\usepackage[utf8]{inputenc}
% ----------------------------------------------------------------------------------------
% Language settings
\usepackage[english]{babel}


% ========================================================================================
% Fonts
% - Helvet is loaded by styles/beamerfonts
% - We use serif for math environements
% - isomath is used for upGreek letters
% ----------------------------------------------------------------------------------------
\usepackage{isomath}
\usepackage{booktabs}
\usefonttheme[onlymath]{serif}
\usepackage{exscale}
\usepackage{anyfontsize}
\setbeamercolor{alerted text}{fg=BaseColor}
% ----------------------------------------------------------------------------------------
% custom commands for symbols
\usepackage{styles/symbols}


% ========================================================================================
% Setup for Titlepage
% ----------------------------------------------------------------------------------------
\title[Textkompressionsalgorithmen in Datenbanken - Eine Forschungsfeldübersicht]{Textkompressionsalgorithmen in Datenbanken - Eine Forschungsfeldübersicht}
\subtitle{Big Data Seminar}
\author[D. Roppelt]{
}
%
\institute[FAU]{%
David Roppelt, Mail: david.roppelt@fau.de \and%
}

% Instead of \institute you can also use the \thanks command
% ------------------------------------------------
%\author[T. Roith]{
%Tim Roith\thanks{Friedrich-Alexander Universität Erlangen-Nürnberg, Department Mathematik}\and%
%Second Author\thanks{Second Insitute}\and%
%Third Author\thanks{Third Insitute}%
%}

\date{February 23, 2026}


% ================================================
% Bibliography
% ------------------------------------------------
\usepackage{csquotes}
\usepackage[style=alphabetic, %alternatively: numeric, numeric-comp, and other from biblatex
			defernumbers=true,
			useprefix=true,%
			giveninits=true,%
			hyperref=true,%
			autocite=inline,%
			maxcitenames=5,%
			maxbibnames=20,%
			uniquename=init,%
			sortcites=true,% sort citations when multiple entries are passed to one cite command
			doi=true,%
			isbn=false,%
			url=false,%
			eprint=false,%
			backend=biber%
		   ]{biblatex}
\addbibresource{bibliography.bib}
\setbeamertemplate{bibliography item}[text]


% ================================================
% Hyperref and setup
% ------------------------------------------------
\usepackage{hyperref}
\hypersetup{
	colorlinks = true,
	final=true, 
	plainpages=false,
	pdfstartview=FitV,
	pdftoolbar=true,
	pdfmenubar=true,
	pdfencoding=auto,
	psdextra,
	bookmarksopen=true,
	bookmarksnumbered=true,
	breaklinks=true,
	linktocpage=true,
	urlcolor=BaseColor,
	citecolor=BaseColor,
	linkcolor=BaseColor
}


% ================================================
% Additional packages
% ------------------------------------------------


% ================================================
% Various custom commands
% ------------------------------------------------
%\setbeameroption{show notes on second screen}
\begingroup\expandafter\expandafter\expandafter\endgroup
\expandafter\ifx\csname pdfsuppresswarningpagegroup\endcsname\relax
\else
  \pdfsuppresswarningpagegroup=1\relax
\fi
% Change color for cite locally
\newcommand{\colorcite}[3]{{\hypersetup{citecolor=#1}{\cite[#2]{#3}}}} 
% ------------------------------------------------
% ================================================
% The main document
% ------------------------------------------------
\begin{document}
% Title page
\begin{frame}[t,titleimage]{-}
\titlepage%
\end{frame}

% Stylized outline
\begin{frame}[title]{-}
	\setcounter{tocdepth}{1}
	\tableofcontents
\end{frame}


\setcounter{tocdepth}{2}

\AtBeginSection[]{
	\begin{frame}[title]{Kapitelübersicht}
		\tableofcontents[
		sections={\thesection},
		subsectionstyle=show/show/hide
		]
	\end{frame}
}

\section{Einleitung}
\subsection{Motivation}
% Introduction
\begin{frame}[t]{Motivation}{Warum wird Datenkompression immer wichtiger?}
	\begin{figure}[H]
		\centering
		\includegraphics[width=0.74\linewidth]{statista2}
		\caption{Volumen der jährlich generierten/replizierten digitalen Datenmenge weltweit von 2010 bis 2024 und Prognose für 2029 \cite{statista_datenvolumen_weltweit}}
		\label{fig:statista}
	\end{figure}
	\footnotetext{\cite{statista_datenvolumen_weltweit} \url{https://de.statista.com/statistik/daten/studie/267974/umfrage/prognose-zum-weltweit-generierten-datenvolumen/}}
\end{frame}
\subsection{Problemstellung}
% ..............................................................................
\begin{frame}[t]{Problemstellung}{Herausforderungen von Kompression im Datenbankkontext}
	\begin{itemize}
		\item Stark wachsendes Datenvolumen in modernen Anwendungen
		\vspace{0.5cm}
		\item Großer Anteil textueller Daten (Logs, JSON, Dokumente, Webdaten)
		\vspace{0.5cm}
		\item Kompression reduziert Speicherbedarf und I/O-Kosten
		\vspace{0.5cm}
		\item Klassische Kompression nicht für Datenbankabfragen optimiert
	\end{itemize}
	
	\vspace{1cm}
	
	\textbf{Zentrale Herausforderung}
	\vspace{0.5cm}
	\begin{itemize}
		\item Hohe Kompressionsrate $\rightarrow$ langsamer Zugriff
		\vspace{0.5cm}
		\item Schneller Zugriff $\rightarrow$ geringere Kompressionsrate
	\end{itemize}
\end{frame}

\begin{frame}[t]{Ziel der Arbeit}{Welche Ziele verfolgt meine Arbeit?}
	\begin{itemize}
		\item Keine Entwicklung eines neuen Kompressionsverfahrens
		\vspace{0.5cm}
		\item Strukturierter Überblick über bestehende Textkompressionsverfahren
		\vspace{0.5cm}
		\item Vergleich anhand einheitlicher Bewertungskriterien
		\vspace{0.5cm}
		\item Einordnung im Kontext verschiedener Datenbankarchitekturen
	\end{itemize}
	
	\vspace{1cm}
	
	\textbf{Forschungsfragen:}
	
	\vspace{0.2cm}
	
	\begin{quote}
		Welches Textkompressionsverfahren ist für welchen Datenbanktyp und welches Anfrageaufkommen geeignet?
	\end{quote}
	\vspace{0.5cm}
	\begin{quote}
		Welche besonderen Eigenschaften besitzen die verschiedenen Verfahren?
	\end{quote}
	\vspace{0.5cm}
	\begin{quote}
		Gibt es ein allgemein bestes Verfahren?
	\end{quote}
\end{frame}

\subsection{Ziel der Arbeit}

\section{Grundlagen}

\subsection{Verlustfreie vs. verlustbehaftete Kompression}
\begin{frame}{Verlustfreie vs. verlustbehaftete Kompression \cite{salomon2004datacompression}}
		
	\begin{columns}[T]
		
		% ---- Links ----
		\column{0.48\textwidth}
		\begin{block}{Verlustfreie Kompression}
			\begin{itemize}
				\item Exakte Rekonstruktion der Originaldaten
				\vspace{0.5cm}
				\item Kein Informationsverlust
				\vspace{0.5cm}
				\item Geringere Kompressionsrate
				\vspace{0.5cm}
				\item Direkter Einsatz in Datenbanken
				\vspace{0.5cm}
				\item Wichtig für Korrektheit der Daten
				\vspace{0.5cm}
				\item Beispiele: ZIP, Huffman, LZ77
			\end{itemize}
		\end{block}
		
		% ---- Rechts ----
		\column{0.48\textwidth}
		\begin{block}{Verlustbehaftete Kompression}
			\begin{itemize}
				\item Nur angenäherte Rekonstruktion
				\vspace{0.5cm}
				\item Entfernt unwichtige Informationen
				\vspace{0.5cm}
				\item Höhere Kompressionsrate möglich
				\vspace{0.5cm}
				\item Für Wahrnehmungsdaten optimiert
				\vspace{0.5cm}
				\item Für Datenbanken (meistens) ungeeignet
				\vspace{0.5cm}
				\item Beispiele: JPEG, MP3, MPEG
			\end{itemize}
		\end{block}		
	\end{columns}
	\footnotetext{\color{BaseColor}\hyperlink{ref}{Sal04}  \color{black}Salomon D. (2004). \emph{Data Compression: The Complete Reference}.}
\end{frame}
\subsection{Prinzipien der Informationsreduktion}
\begin{frame}{Prinzipien der Informationsreduktion \cite{salomon2004datacompression}}
	
	\begin{block}{Entropie}
		\begin{itemize}
			\item Maß für die durchschnittliche Informationsmenge einer Datenquelle
			\item Definiert die theoretische Untergrenze der Kompression
			\item Hohe Entropie $\rightarrow$ geringe Komprimierbarkeit
		\end{itemize}
	\end{block}
	
	\begin{block}{Redundanz}
		\begin{itemize}
			\item Wiederkehrende oder vorhersehbare Strukturen in Daten
			\item Kann entfernt werden, ohne Information zu verlieren
			\item Grundlage jeder verlustfreien Kompression
		\end{itemize}
	\end{block}
	
	\begin{block}{Kodierung}
		\begin{itemize}
			\item Systematische Zuordnung von Symbolen zu Codewörtern
			\item Häufige Symbole erhalten kürzere Codes
			\item Muss eindeutig dekodierbar sein
		\end{itemize}
	\end{block}
	\footnotetext{\color{BaseColor}\hyperlink{ref}{Sal04}  \color{black}Salomon D. (2004). \emph{Data Compression: The Complete Reference}.}
\end{frame}

\subsection{Modellierungsansätze der Kompression}
%weglassen?
\subsubsection{Statistische Verfahren}
\subsubsection{Wörterbuchbasierte Verfahren}

\subsection{Bewertungskriterien von Kompressionsverfahren}
	\begin{frame}{Bewertungskriterien von Kompressionsverfahren}
		\begin{block}{Kompressionsrate}
			\begin{itemize}
				\item Verhältnis zwischen Originalgröße und komprimierter Größe
				\item Maß für Speicherersparnis
			\end{itemize}
		\end{block}
		
		\begin{block}{Kompressions- und Dekompressionsgeschwindigkeit}
			\begin{itemize}
				\item Laufzeit beim Komprimieren und Dekomprimieren
				\item Entscheidend für Abfrageperformance
			\end{itemize}
		\end{block}
		
		\begin{block}{Speichermehraufwand}
			\begin{itemize}
				\item Zusätzlicher Speicher für Metadaten oder Wörterbücher
				\item Einfluss auf Gesamteffizienz
			\end{itemize}
		\end{block}
		
		\begin{block}{Direkter Zugriff}
			\begin{itemize}
				\item Möglichkeit, auf einzelne Werte ohne vollständige Dekompression zuzugreifen
				\item Besonders relevant für analytische Workloads
			\end{itemize}
		\end{block}
	\end{frame}
\subsection{Datenbankarchitekturen}
\begin{frame}{Datenbankarchitekturen \cite{abadi2008columnstores, kaufmann2023sqlnosql}}
	
	\begin{columns}[T,totalwidth=\textwidth]
		
		% ---- Linke Spalte ----
		\column{0.48\textwidth}
		
		\begin{block}{Relationale Datenbanken (Row Store)}
			\begin{itemize}
				\item Speicherung zeilenweise
				\item Optimiert für Transaktionen (OLTP)
				\item Schneller Zugriff auf vollständige Tupel
				\item Kompression meist sekundär
			\end{itemize}
		\end{block}
		
		\begin{block}{Spaltenorientierte Datenbanken (Column Store)}
			\begin{itemize}
				\item Speicherung spaltenweise
				\item Optimiert für analytische Abfragen (OLAP)
				\item Hohe Komprimierbarkeit durch ähnliche Werte
				\item Kompression zentraler Bestandteil
			\end{itemize}
		\end{block}
		
		% ---- Rechte Spalte ----
		\column{0.48\textwidth}
		
		\begin{block}{NoSQL-Datenbanken}
			\begin{itemize}
				\item Flexible Datenmodelle (z.\,B. Dokumente, Key-Value)
				\item Hohe Skalierbarkeit
				\item Oft generische Kompressionsverfahren
				\item Fokus auf Verfügbarkeit und Verteilung
			\end{itemize}
		\end{block}
		
	\end{columns}
	
	\vspace{0.3cm}
	
	\textbf{Relevanz:}  
	Die Datenbankarchitektur beeinflusst maßgeblich die Wahl des geeigneten Kompressionsverfahrens.
	\footnotetext{\color{BaseColor}\hyperlink{ref}{AMH08}  \color{black}Abadi et al. (2008). \emph{Column-stores vs. Row-stores: How Different Are They Really?}.}
	\footnotetext{\color{BaseColor}\hyperlink{ref}{KM23}  \color{black}Kaufmann M. und Meier A. (2023). \emph{SQL- \& NoSQL-Datenbanken}.}
\end{frame}


\section{Analyse ausgewählter Kompressionsverfahren}

\subsection{FSST + GSST}
\begin{frame}[fragile]{FSST: Fast Random Access String Compression \cite{fsst}}
	\begin{columns}[T]
		\column{0.48\textwidth}
		\begin{itemize}
			\item Leichtgewichtiges, verlustfreies Kompressionsverfahren für Zeichenketten
			\item Entwickelt speziell für spaltenorientierte Datenbanksysteme
			\item Verwendet eine statische Symboltabelle (\textbf{F}ast \textbf{S}tatic \textbf{S}ymbol \textbf{T}able)
			\item Häufige Byte-Sequenzen werden durch kurze Codes ersetzt
		\end{itemize}
		
		\vspace{0.4cm}
		
		\textbf{Zentrale Eigenschaften}
		\begin{itemize}
			\item Sehr schnelle Dekompression
			\item Direkter Zugriff auf einzelne Blöcke möglich
			\item Geringer Speichermehraufwand
			\item Gute Balance zwischen Speicherreduktion und Laufzeit
			\item Nutzung moderner CPU-Vektorisierung (SIMD)
		\end{itemize}
		
		\column{0.48\textwidth}
		\begin{figure}[H]
			\centering
			\includegraphics[width=\linewidth]{figures/fsst}
			\caption{Beispielhafte Wörterbuchkompression eines URL-Korpus. Nachbau aus \cite{fsst}}
			\label{fig:fsst}
		\end{figure}
	\end{columns}
	\footnotetext{\color{BaseColor}\hyperlink{ref}{BNL20}  \color{black}Boncz, P.; Neumann, T.; Leis, V. (2020). \emph{FSST: Fast Random Access String Compression}.}
\end{frame}

\begin{frame}{GSST: Parallel string decompression at 191 GB/s on GPU \cite{10.1145/3719330.3721228}}
	\begin{itemize}
		\item Erweiterung von FSST für GPU-basierte Systeme
		\item Kein neues Kompressionsverfahren, sondern parallele Dekompression
		\item Nutzt die statische Symboltabelle von FSST
	\end{itemize}
	
	\vspace{0.4cm}
	
	\textbf{Zentrale Idee}
	\begin{itemize}
		\item Aufteilung der Daten in unabhängige Blöcke
		\item Parallele Verarbeitung durch viele GPU-Threads
		\item Hohe Dekompressionsbandbreite
	\end{itemize}
	
	\vspace{0.4cm}
	
	\textbf{Eigenschaften}
	\begin{itemize}
		\item Sehr hoher Durchsatz bei analytischen Workloads
		\item Besonders dann sinnvoll, wenn GPU verfügbar ist
		\item Zeigt Bedeutung hardwareangepasster Kompressionsformate
	\end{itemize}
	\footnotetext{\color{BaseColor}VHA25  \color{black}Vonk, R; Hoozemans, J.; Al-Ars, Z. (2025). \emph{GSST: Parallel string decompression at 191 GB/s on GPU}.}
\end{frame}

\subsection{DOPSC}
{
\begin{frame}{\large Dictionary-based order-preserving string compression for main memory column stores (DOPSC) \cite{DOPSC}}
	
	\begin{itemize}
		\item Wörterbuchbasiertes, verlustfreies Kompressionsverfahren
		\item Entwickelt für spaltenorientierte Datenbanksysteme
		\item Fokus auf effiziente Vergleichs- und Sortieroperationen
	\end{itemize}
	
	\vspace{0.4cm}
	
	\textbf{Zentrale Idee}
	\begin{itemize}
		\item Erhaltung der lexikographischen Ordnung
		\item Vergleiche direkt auf komprimierten Daten möglich
		\item Keine vollständige Dekompression notwendig
	\end{itemize}
	
	\vspace{0.4cm}
	
	\textbf{Vorteile}
	\begin{itemize}
		\item Effiziente Intervallabfragen
		\item Schnelles Sortieren
		\item Gute Integration in spaltenorienterte Datenbanken
	\end{itemize}
	
	\vspace{0.4cm}
	
	\textbf{Einordnung}
	\begin{itemize}
		\item Kompromiss zwischen Kompressionsrate und Zugriffsgeschwindigkeit
		\item Besonders geeignet für analytische Workloads
	\end{itemize}
	
	\footnotetext{\color{BaseColor}BHF09  \color{black}Binnig et al. (2025). \emph{Dictionary-based order-preserving string compression for main memory column stores}.}
\end{frame}
}


\subsection{Fast \& Strong}
	\begin{frame}[fragile]{Fast \& Strong: The Case of Compressed String Dictionaries on Modern CPUs \cite{fast}}
	\begin{columns}[T]
		\column{0.48\textwidth}
		\begin{itemize}
			\item Fokus auf höhere Kompressionsrate in spaltenorientierten Datenbanken
			\item Untersuchung des Zielkonflikts zwischen Kompression und Laufzeit
			\item Kombination mehrerer Kompressionstechniken (mehrstufige Pipeline)
		\end{itemize}
		
		\vspace{0.4cm}
		
		\textbf{Zentrale Idee}
		\begin{itemize}
			\item Stärkere Kompression reduziert I/O-Kosten
			\item Zusätzliche Dekompressionskosten können durch weniger Datenzugriffe kompensiert werden
			\item Ebenfalls Nutzung moderner CPU-Vektorisierung (SIMD)
		\end{itemize}
		
		\vspace{0.4cm}
		
		\textbf{Erkenntnis}
		\begin{itemize}
			\item Mehr Kompression bedeutet nicht automatisch schlechtere Abfrageperformance
			\item Optimale Wahl hängt vom Workload ab
		\end{itemize}
		
		\column{0.48\textwidth}
		\begin{figure}[H]
			\centering
			\includegraphics[width=\linewidth]{figures/fast_mt_pfeilen}
			\caption{Beispiel von PFC und RPFC. Nachbau aus \cite{fast}}
			\label{fig:fsst}
		\end{figure}
	\end{columns}
	\footnotetext{\color{BaseColor}\hyperlink{ref}{Las+19}  \color{black}Lasch et al. (2019). \emph{Fast \& Strong: The Case of Compressed String Dictionaries on Modern CPUs}.}
	\end{frame}
\subsection{AlphaZip}

\subsection{OnPair}

\section{Vergleich und Einordnung}

\subsection{Vergleich anhand der Bewertungskriterien}

\subsection{Zentrale Zielkonflikte}

\section{Evaluation}

\subsection{Kernerkenntnisse}
\begin{frame}[t]{Kernerkenntnisse}{}
\end{frame}

\subsection{Limitationen}
\begin{frame}[t]{Limitationen}{}
\end{frame}

\subsection{Ausblick}
\begin{frame}[t]{Ausblick}{}
\end{frame}

\begin{frame}[t,titleimage]{-}
	\titlepage%
\end{frame}

\begin{frame}[allowframebreaks]{Referenzen}
	\label{ref}
	\printbibliography[heading=none]
\end{frame}
%
%
%
%
%
\end{document}